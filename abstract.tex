% Thesis Abstract --------------------------------------------------------------

\prefacesection{Abstract}

\begin{comment}
Despite the fact that an abstract is quite brief, it must do almost as much work as the multi-page
paper that follows it. In a computer science paper, this means that it should in most cases include
the following sections. Each section is typically a single sentence, although there is room for
creativity. In particular, the parts may be merged or spread among a set of sentences. Use the
following as a checklist for your next abstract (URL:
http://www.ece.cmu.edu/~koopman/essays/abstract.html):
\end{comment}

The main goal of this project will be to develop a secure web-based exam paper generation system. The system should support multiple users, subjects and courses. The end product should allow for the generation of pools of questions within a topic area and should also be able to generate exam papers based on a random selection of questions within a given pool or subject area. \\ The system should store historical data about which questions were used in what year and should avoid using the same question year on year. The system should also allow the user to have the ultimate say in what questions are chosen by allowing randomly selected questions to be replaced with other questions from with the same pool. \\ The system should also provide the ability to link marking schemes with questions and should ensure marking schemes are correct and add up to the correct totals. The system should be entirely secure from the login process to the storage of data and data in transit. \\ \\
Keywords: Web-frameworks, Security, Encryption, Databases, HTML, CSS, Java

\begin{description}
  \item[Motivation:]
  
  \begin{comment}
Why do we care about the problem and the results? If the problem isn't obviously
"interesting" it might be better to put motivation first; but if your work is incremental progress
on a problem that is widely recognised as important, then it is probably better to put the problem
statement first to indicate which piece of the larger problem you are breaking off to work on. This
section should include the importance of your work, the difficulty of the area, and the impact it
might have if successful.
\end{comment}

With the growth of mature students entering third level educational institutions around Ireland. Mainly due to the economic downturn which has resulted in many working class people finding themselves unemployed. \\It is for this reason why programmes were put into place by the Irish Government and European Union to establish a structure by investing in these peoples education and skills and thus offering them support when it comes to seeking employment once again. Either in fields which they are familiar with or something completely different. \\ \\It is initiatives like these which is supported and funded by authorities and governing bodies such as SUSI and the ESF Ireland which have lead to bringing the unemployment rate of persons aged 25 - 74 years down in the last two years by 2.7\% \cite{Cso2016} \cite{Esf2016}. \\The difference between the total number of people now employed since September 2014 and September 2016 is 50, 200 persons. That is a great deal of people in terms of a two year period. These people attending the institutions could have not seen a classroom for more than twenty to thirty or up to forty years or more let alone used a computer and thus require a greater deal of help or attention. As some of them might be exploring an IT field. \\It is this one on one attention that they desperately need and freeing up a lecturers time for this would be of great benefit. Since being exposed to college life and the reality of being a mature student have witnessed this firsthand. Most lecturers teach numerous subjects. And for each subject they need to put together numerous exam papers to account for semesters and repeats. This could total many hours of composition for their many subjects. \\If a system was in place to automatically generate exam papers this will take away from the time that it takes to compose them. They would be able to give this time back to their students. In the form of a meeting for Q \& A or for revision work. This is the impact that an exam paper generation system could have on the lives of mature students if successful.

  \item[Problem statement:]

  \begin{comment}  
What problem are you trying to solve? What is the
scope of your work (a generalised approach, or for a specific situation)? Be careful not to use too
much jargon. In some cases it is appropriate to put the problem statement before the motivation,
but usually this only works if most readers already understand why the problem is important.
\end{comment}

Compiling an exam paper which will test the students knowledge in a particular subject is challenging in its own way. Firstly, there are the time constraints. Examiners a being faced with more and more work each year within the same space of time. If you add up the amount of time per semester over a given year which is set aside to put together an exam paper it will add up to many hours. \\There needs to be a better approach to minimise these hours. It will take a great deal of time to produce a good quality paper. The questions need to be taken from the curriculum which was or is being delivered to the students over the semester. This brings upon the need to develop a paper from as many of the important areas of the module as possible. As this will be the process of determining the students performance with regard to the questions which are asked and the complexity of them. The result of good exam question will determine the sort of student the college will produce.

  \item[Approach:]
  
    \begin{comment}
How did you go about solving or making progress on the problem? Did you use simulation,
analytic models, prototype construction, or analysis of field data for an actual product? What was
the extent of your work (did you look at one application program or a hundred programs in twenty
different programming languages?) What important variables did you control, ignore, or measure?
\end{comment}

The approach towards this project will be to do as much research as possible around the work of others. Reading the research papers which are available on the internet and in journals. \\Which technologies and methodologies were used and incorporated into their work. If they had any shortcomings. See which improvements can be made. One can streamline a complicated version if one is available. And combining the methods of others to form one which is more successful. Furthermore, how long ago their work took place as perhaps a technology has caught up to what they were trying to achieve and would now be in a position to overcome any weaknesses. \\For the purpose of this project the SDLC of choice will be the Agile Model. Which takes measures such as planning, analysis, designing building and testing. This will be done in small increments. Each time building on what is currently available and adding to it. Until all features are in place for a full system release.

  \item[Results:]
  
  \begin{comment}
What's the answer? Specifically, most good computer architecture papers conclude that
something is so many percent faster, cheaper, smaller, or otherwise better than something else. Put
the result there, in numbers. Avoid vague, hand-waving results such as "very", "small", or
"significant." If you must be vague, you are only given license to do so when you can talk about
orders-of-magnitude improvement. There is a tension here in that you should not provide numbers
that can be easily misinterpreted, but on the other hand you don't have room for all the caveats.
\end{comment}

A conclusion has yet to be established. The results should represent a system which is better than the one in place offering a defined gain to the current method for examination compilation. This will be determined once the system is in place and can be tested in a scientific manner by comparing the two methods.

  \item[Conclusions:]
  
    \begin{comment}
What are the implications of your answer? Is it going to change the world (unlikely),
be a significant "win", be a nice hack, or simply serve as a road sign indicating that this path is
a waste of time (all of the previous results are useful). Are your results general, potentially
generalisable, or specific to a particular case?
\end{comment}

To follow.

\end{description}




\smallskip


% ------------------------------------------------------------------------------
