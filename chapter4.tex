\chapter{Implementation - "Building the solution"}


\section{Introduction}

This chapter is a walkthrough of the steps which describe the construction of the application. 

So it is important to understand the OO analysis and design concepts. Now the most important purpose of OO analysis is to identify objects of a system to be designed. This analysis is also done for an existing system. Now an efficient analysis is only possible when we are able to start thinking in a way where objects can be identified. After identifying the objects their relationships are identified and finally the design is produced. 

So the purpose of OO analysis and design can described as:
\begin{itemize}
    \item Identifying the objects of a system.

    \item Identify their relationships.

    \item Make a design which can be converted to executables using OO languages.
\end{itemize}
There are three basic steps where the OO concepts are applied and implemented. The steps can be defined as

OO Analysis - OO Design - OO implementation using OO languages

Now the above three points can be described in details:

    During object oriented analysis the most important purpose is to identify objects and describing them in a proper way. If these objects are identified efficiently then the next job of design is easy. The objects should be identified with responsibilities. Responsibilities are the functions performed by the object. Each and every object has some type of responsibilities to be performed. When these responsibilities are collaborated the purpose of the system is fulfilled.

    The second phase is object oriented design. During this phase emphasis is given upon the requirements and their fulfilment. In this stage the objects are collaborated according to their intended association. After the association is complete the design is also complete.

    The third phase is object oriented implementation. In this phase the design is implemented using object oriented languages like Java, C++ etc.

\subsection{OO Analysis and Design}
Brief discussion of how the analysis brings to a design
\subsection{Role of UML in OO design}
Modeling the software and understanding the relationships between the OO design and UML
\subsection{UML Diagrams}
UML diagrams used, such as use class, interface, collaboration, use case and component


\section{System Design}

\subsection{ERD Creation}
Formulating how the entities will interact
\subsection{Table Creation}
Providing a table of entities to be executed
\subsection{Test Tables}
Running the tables in Netbeans

\section{IDE Choice}

\subsection{Sublime}
Pros and Cons
\subsection{Netbeans}
Pros and Cons
\subsection{Eclipse}
Pros and Cons

\section{Appearance}

\subsection{Wire Framing}
Wire framing, what the interface should look like
\subsection{CSS}
CSS to achieve the Wire Framing design
\subsection{HTML}
HTML to implement the format

\section{Technologies}

\subsection{Netbeans IDE}
NetBeans IDE make it easy to create Java EE based Web application projects with either JSF 2.2 (Facelets), JSP's or Servlets. In addition, you can can create and work with Web applications using other frameworks like the Spring, Struts, and Hibernate frameworks.
\subsection{Server}
Talk about GlassFish
\subsection{Framework}
JavaServer Pages (JSP) is a technology that helps software developers create dynamically generated web pages based on HTML, XML, or other document types. Released in 1999 by Sun Microsystems, JSP is similar to PHP and ASP, but it uses the Java programming language.
\subsection{Java OO}
What kind of Java Programming will be used
\subsection{Encryption}
Types of encryption
\subsection{BLOB}
BLOB (binary large object) How to store the data in the tables to accomodate engineering papers
\subsection{Database Choice}
PostgreSQL

